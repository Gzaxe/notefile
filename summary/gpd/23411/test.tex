\documentclass[tikz,border=10pt]{standalone}
\usepackage{tikz}
\usetikzlibrary{positioning}
\usepackage{tikz-feynman}
\begin{document}

\begin{tikzpicture}
	\begin{feynman}
	%% fig c
	\vertex(j1) at (0,0);
	\vertex[right =1cm  of j1] (j2);
	\vertex[right =2cm  of j1, crossed dot,anchor=center] (j3) {};
	\vertex[right =3cm  of j1] (j4);
	\vertex[right =4cm  of j1] (j5);
	\vertex[above =1.4 cm  of j3] (j6);
	\node[above =0.5 cm  of j5] {$j$};
	%% fig d
	\vertex[above right =0 cm and 5 cm of j1] (d1);
	\vertex[right =1cm  of d1] (d2);			
	\vertex[right =2cm  of d1,square dot,anchor=center] (d3) {};
	\vertex[right =3cm  of d1] (d4);
	\vertex[right =4cm  of d1] (d5);     
	\vertex[above =1.4 cm  of d3] (d6);
	\node[above =0.5 cm  of d5] {$d$};
	%% fig e
	\vertex[above right =0 cm and 5 cm of d1] (e1);
	\vertex[right =1cm  of e1] (e2);
	\vertex[right =2cm  of e1,dot,anchor=center] (e3) {};
	\vertex[right =3cm  of e1] (e4);
	\vertex[right =4cm  of e1] (e5);
	\vertex[above =1.4 cm  of e3] (e6);
	\node[above =0.5 cm  of e5] {$e$};
	%% fig a
	\vertex[above right =2.5 cm and 0 cm of j1] (a1);
	\vertex[right =1cm  of a1] (a2);
	\vertex[right =2cm  of a1] (a3);
	\vertex[right =3cm  of a1] (a4);
	\vertex[right =4cm  of a1] (a5);
	\vertex[above =1.4 cm  of a3,crossed dot,anchor=center] (a6){};
	\node[above =0.5 cm  of a5] {$a$};
	%% fig b
	\vertex[above right = 0 cm and 5 cm of a1] (b1);
	\vertex[right =1cm  of b1] (b2);
	\vertex[right =2cm  of b1,square dot,fill=gray,anchor=center] (b3) {};
	\vertex[right =3cm  of b1] (b4);
	\vertex[right =4cm  of b1] (b5);
	\vertex[above =1.4 cm  of b3, crossed dot, anchor=center] (b6){};
	\node[above =0.5 cm  of b5] {$b$};
	
		% 对各个顶点连线
		\diagram*{
		{ [edge= fermion]
				(b1) -- (b3)--(b5),
			(a1) -- (a3)--(a5),
			(d1) -- (d3)--(d5), 
			(e1) -- (e3)--(e5), 
			(j1) -- (j3)--(j5),
			},
		%介子连线
		{ [edge= charged scalar]
		 (b3) --[half left](b6)--[half left](b3),
		(a3) --[half left](a6)--[half left](a3),
		(e3) --[half left](e6)--[half left](e3),
		(d3) --[half left](d6)--[half left](d3),
		(j3) --[half left](j6)--[half left](j3),
		}
		};
	\end{feynman}
\end{tikzpicture}


\end{document}