\documentclass[tikz,border=10pt]{standalone}
\usepackage{tikz}
\usetikzlibrary{positioning}
\usepackage{tikz-feynman}
\begin{document}
	
	\begin{tikzpicture}[
		decuplet/.style={ % 自定义一个重子的双线
			double distance=1pt,
			postaction={decorate}, decoration={
				markings, mark=at position .6 with {
					\arrow{Triangle[angle=40:2pt 3]}
				},
			}
		}
		]
		\begin{feynman}
				%% fig b
			\vertex(b1) at (0,0);
			\vertex[right =1cm  of b1] (b2);
			\vertex[right =2cm  of b1, crossed dot,anchor=center] (b3){};
			\vertex[right =3cm  of b1] (b4);
			\vertex[right =4cm  of b1] (b5);
			\node[above =0.5cm of b5] {$b$};
			%% fig left
			\vertex[above right =1.5 cm and 0 cm of b1] (l1);
			\vertex[right =0.5cm  of l1, crossed dot,anchor=center] (l2){};
			\vertex[right =1cm  of l1] (l3);
			\vertex[right =3cm  of l1] (l4);
			\vertex[right =4cm  of l1] (l5);
			\node[above =0.5cm of l5] {$1$};
            %%fig right
            \vertex[above right =0 cm and 5 cm of l1] (r1);
            \vertex[right =1cm  of r1] (r2);
            \vertex[right =3cm  of r1] (r3);
            \vertex[right =3.5cm  of r1, crossed dot,anchor=center] (r4){};
            \vertex[right =4cm  of r1] (r5);
            \node[above =0.5cm of r5] {$2$};			
				% 对各个顶点连线
			\diagram*{
				{
					[edge=plain]
				 (b1)--(b2)--(b3)--(b4)--(b5),	
				(l1)--(l2)--(l3)--(l4)--(l5),
				(r1)--(r2)--(r3)--(r4)--(r5),	
				},
				% 介子连线
				{
					[edge= charged scalar]
				(b2) --[half left](b4),
				(l3) --[half left](l4),
                (r2) --[half left](r3),
				}
			};
			%% 添加重子双线
			
		\end{feynman}
	\end{tikzpicture}

\end{document}
