\documentclass[tikz,border=10pt]{standalone}
\usepackage{tikz}
\usetikzlibrary{positioning}
\usepackage{tikz-feynman}
\begin{document}
	
	\begin{tikzpicture}[
		decuplet/.style={ % 自定义一个重子的双线
			double distance=1pt,
			postaction={decorate}, decoration={
				markings, mark=at position .6 with {
					\arrow{Triangle[angle=40:2pt 3]}
				},
			}
		}
		]
		\begin{feynman}
	     
	     %% fig h
	     \vertex(h1) at (0,0);
	     \vertex[right =1cm  of h1] (h2);
	     \vertex[right =2cm  of h1, crossed dot,anchor=center] (h3) {};
	     \vertex[right =3cm  of h1] (h4);
	     \vertex[right =4cm  of h1] (h5);
	     \vertex[above =1.4 cm  of h3] (h6);
	     \node[above =0.5 cm  of h5] {$tadpole$};
	     
	     	%% fig n
	     \vertex[above right =2.5 cm and 0 cm of h1] (n1);
	     \vertex[right =1cm  of n1] (n2);
	     \vertex[right =2cm  of n1,crossed dot,anchor=center] (n3){};
	     \vertex[right =3cm  of n1] (n4);
	     \vertex[right =4cm  of n1] (n5);
	     \node[above =0.5 of n5] {$rbw-dec$};
	        %% fig p
			\vertex[above right =0 cm and 5 cm of n1] (p1);
			\vertex[right =1cm  of p1] (p2);
			\vertex[right =2cm  of p1,square dot,anchor=center] (p3){};
			\vertex[right =3cm  of p1] (p4);
			\vertex[right =4cm  of p1] (p5);
			\node[above =0.5 of p5] {$tran-left$};
			%% fig q
			\vertex[above right =0 cm and 5 cm of p1] (q1);
			\vertex[right =1cm  of q1] (q2);
			\vertex[right =2cm  of q1,square dot,anchor=center] (q3){};
			\vertex[right =3cm  of q1] (q4);
			\vertex[right =4cm  of q1] (q5);
			\node[above =0.5 of q5] {$tran-right$};
		
			%% fig b
		\vertex[above right =2.5 cm and 0 cm of n1] (b1);
		\vertex[right =1cm  of b1] (b2);
		\vertex[right =2cm  of b1, crossed dot,anchor=center] (b3){};
		\vertex[right =3cm  of b1] (b4);
		\vertex[right =4cm  of b1] (b5);
		\node[above =0.5cm of b5] {$rbw-oct$};
		
			%% fig d
			\vertex [above right =0 cm and 5 cm of b1] (d1);
			\vertex[right =1cm  of d1] (d2);
			\vertex[right =2cm  of d1] (d3);
			\vertex[right =3cm  of d1, crossed dot,anchor=center] (d4){};
			\vertex[right =4cm  of d1] (d5);
			\node[above =0.5 cm  of d5] {$kr$};
				%% fig e
			\vertex[above right =0 cm and 5 cm of d1] (e1);
			\vertex[right =1cm  of e1, crossed dot,anchor=center] (e2){};
			\vertex[right =2cm  of e1] (e3);
			\vertex[right =3cm  of e1] (e4);
			\vertex[right =4cm  of e1] (e5);
			\node[above =0.5cm of e5] {$kr$};
			
			
		
			
			% 对各个顶点连线
			\diagram*{
				{
					[edge=plain]
					(h1) -- (h3)--(h5),
					(d1) -- (d2)--(d4)-- (d5),
				    (e1) -- (e2)--(e4)-- (e5),
			       (n1)-- (n2) -- (n3)--(n4)--(n5),	
                    (p1)-- (p2) -- (p3)--(p4)--(p5),	
                    (q1)-- (q2) -- (q3)--(q4)--(q5),
                    (b1)--(b2)--(b3)--(b4)--(b5),	
                   	},
				% 介子连线
				{
					[edge= charged scalar]
				(h3) --[half left](h6)--[half left](h3),
					(d2) --[half left](d4),
					(e2) --[half left](e4),
					(n2) --[half left](n4),
					(p2) --[half left](p4),
					(q2) --[half left](q4),
				    (b2) --[half left](b4),
				   	}
			};
			%% 添加重子双线
			\draw[decuplet]  (n2) -- (n3);
			\draw[decuplet]  (n3) -- (n4);
		    \draw[decuplet]  (p2) -- (p3);
			\draw[decuplet]  (q3) -- (q4);
		
		\end{feynman}
	\end{tikzpicture}
	
\end{document}
