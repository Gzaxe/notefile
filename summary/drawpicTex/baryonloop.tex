\documentclass[tikz,border=10pt]{standalone}
\usepackage{tikz}
\usetikzlibrary{positioning}
\usepackage{tikz-feynman}
\begin{document}
	
	\begin{tikzpicture}[
		decuplet/.style={ % 自定义一个重子的双线
			double distance=1pt,
			postaction={decorate}, decoration={
				markings, mark=at position .6 with {
					\arrow{Triangle[angle=40:2pt 3]}
				},
			}
		}
		]
		\begin{feynman}
			 %% fig p
			\vertex(p1) at (0,0);
			\vertex[right =1cm  of p1] (p2);
			\vertex[right =2cm  of p1,square dot,anchor=center] (p3){};
			\vertex[right =3cm  of p1] (p4);
			\vertex[right =4cm  of p1] (p5);
			\node[above =0.5 of p5] {$p$};
			%% fig q
			\vertex[above right =0 cm and 5 cm of p1] (q1);
			\vertex[right =1cm  of q1] (q2);
			\vertex[right =2cm  of q1,square dot,anchor=center] (q3){};
			\vertex[right =3cm  of q1] (q4);
			\vertex[right =4cm  of q1] (q5);
			\node[above =0.5 of q5] {$q$};
				%% fig n
			\vertex[above right =2.5 cm and 0 cm of p1] (n1);
			\vertex[right =1cm  of n1] (n2);
			\vertex[right =2cm  of n1,crossed dot,anchor=center] (n3){};
			\vertex[right =3cm  of n1] (n4);
			\vertex[right =4cm  of n1] (n5);
			\node[above =0.5 of n5] {$n$};
			%% fig o
			\vertex[above right =0 cm and 5 cm of n1] (o1);
			\vertex[right =1cm  of o1] (o2);
			\vertex[right =2cm  of o1,square dot,anchor=center] (o3){};
			\vertex[right =3cm  of o1] (o4);
			\vertex[right =4cm  of o1] (o5);
			\node[above =0.5 of o5] {$o$};
				%% fig b
			\vertex[above right =2.5 cm and 0 cm of n1] (b1);
			\vertex[right =1cm  of b1] (b2);
			\vertex[right =2cm  of b1, crossed dot,anchor=center] (b3){};
			\vertex[right =3cm  of b1] (b4);
			\vertex[right =4cm  of b1] (b5);
			\node[above =0.5cm of b5] {$b$};
			%% fig c
			\vertex[above right =0 cm and 5 cm of b1] (c1);
			\vertex[right =1cm  of c1] (c2);
			\vertex[right =2cm  of c1, square dot,anchor=center] (c3){};
			\vertex[right =3cm  of c1] (c4);
			\vertex[right =4cm  of c1] (c5);
			\node[above =0.5cm of c5] {$c$};
			
				% 对各个顶点连线
			\diagram*{
				{
					[edge=plain]
					(n1)-- (n2) -- (n3)--(n4)--(n5),	
					(o1)-- (o2) -- (o3)--(o4)--(o5),
					(p1)-- (p2) -- (p3)--(p4)--(p5),	
					(q1)-- (q2) -- (q3)--(q4)--(q5),
					 (b1)--(b2)--(b3)--(b4)--(b5),	
					(c1)--(c2)--(c3)--(c4)--(c5),	
				},
				% 介子连线
				{
					[edge= charged scalar]
			 	(n2) --[half left](n4),
				(o2) --[half left](o4),
				(p2) --[half left](p4),
				(q2) --[half left](q4),
				(b2) --[half left](b4),
				(c2) --[half left](c4),
				}
			};
			%% 添加重子双线
			\draw[decuplet]  (n2) -- (n3);
			\draw[decuplet]  (n3) -- (n4);
			\draw[decuplet]  (o2) -- (o3);
			\draw[decuplet]  (o3) -- (o4);
			\draw[decuplet]  (p2) -- (p3);
			\draw[decuplet]  (q3) -- (q4);
		\end{feynman}
	\end{tikzpicture}

\end{document}
